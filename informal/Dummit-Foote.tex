\documentclass{article}

\title{\textbf{
Exercises from \\
\textit{Abstract Algebra} \\
by David Dummit and Richard Foote
}}

\date{}

\usepackage{amsmath, amsthm}
\usepackage{amssymb}


\theoremstyle{definition}
\newtheorem{lem}{Lemma}[section]
\newtheorem{cor}[lem]{Corollary}
\newtheorem{prop}[lem]{Proposition}
\newtheorem{thm}[lem]{Theorem}
\newtheorem{remark}[lem]{Remark}
\newtheorem{defn}[lem]{Definition}

\newtheorem*{prop*}{Proposition}
\newtheorem*{thm*}{Theorem}
\newtheorem*{defn*}{Definition}
\newtheorem*{lem*}{Lemma}

\begin{document}
\maketitle

\paragraph{Exercise 1.1.2a} Prove the the operation $\star$ on $\mathbb{Z}$ defined by $a\star b=a-b$ is not commutative.
\begin{proof}
    Not commutative since
$$
1 \star(-1)=1-(-1)=2
$$
$$
(-1) \star 1=-1-1=-2 .
$$
\end{proof}


\paragraph{Exercise 1.1.3} Prove that the addition of residue classes $\mathbb{Z}/n\mathbb{Z}$ is associative.
\begin{proof}
    We have
$$
\begin{aligned}
(\bar{a}+\bar{b})+\bar{c} &=\overline{a+b}+\bar{c} \\
&=\overline{(a+b)+c} \\
&=\overline{a+(b+c)} \\
&=\bar{a}+\overline{b+c} \\
&=\bar{a}+(\bar{b}+\bar{c})
\end{aligned}
$$
since integer addition is associative.
\end{proof}


\paragraph{Exercise 1.1.4} Prove that the multiplication of residue class $\mathbb{Z}/n\mathbb{Z}$ is associative.
\begin{proof}
    We have
$$
\begin{aligned}
(\bar{a} \cdot \bar{b}) \cdot \bar{c} &=\overline{a \cdot b} \cdot \bar{c} \\
&=\overline{(a \cdot b) \cdot c} \\
&=\overline{a \cdot(b \cdot c)} \\
&=\bar{a} \cdot \overline{b \cdot c} \\
&=\bar{a} \cdot(\bar{b} \cdot \bar{c})
\end{aligned}
$$
since integer multiplication is associative.
\end{proof}


\paragraph{Exercise 1.1.5} Prove that for all $n>1$ that $\mathbb{Z}/n\mathbb{Z}$ is not a group under multiplication of residue classes.
\begin{proof}
    Note that since $n>1, \overline{1} \neq \overline{0}$. Now suppose $\mathbb{Z} /(n)$ contains a multiplicative identity element $\bar{e}$. Then in particular,
$$
\bar{e} \cdot \overline{1}=\overline{1}
$$
so that $\bar{e}=\overline{1}$. Note, however, that
$$
\overline{0} \cdot \bar{k}=\overline{0}
$$
for all k, so that $\overline{0}$ does not have a multiplicative inverse. Hence $\mathbb{Z} /(n)$ is not a group under multiplication.
\end{proof}


\paragraph{Exercise 1.1.15} Prove that $(a_1a_2\dots a_n)^{-1} = a_n^{-1}a_{n-1}^{-1}\dots a_1^{-1}$ for all $a_1, a_2, \dots, a_n\in G$.
\begin{proof}
    For $n=1$, note that for all $a_1 \in G$ we have $a_1^{-1}=a_1^{-1}$.
Now for $n \geq 2$ we proceed by induction on $n$. For the base case, note that for all $a_1, a_2 \in G$ we have
$$
\left(a_1 \cdot a_2\right)^{-1}=a_2^{-1} \cdot a_1^{-1}
$$
since
$$
a_1 \cdot a_2 \cdot a_2^{-1} a_1^{-1}=1 .
$$
For the inductive step, suppose that for some $n \geq 2$, for all $a_i \in G$ we have
$$
\left(a_1 \cdot \ldots \cdot a_n\right)^{-1}=a_n^{-1} \cdot \ldots \cdot a_1^{-1} .
$$
Then given some $a_{n+1} \in G$, we have
$$
\begin{aligned}
\left(a_1 \cdot \ldots \cdot a_n \cdot a_{n+1}\right)^{-1} &=\left(\left(a_1 \cdot \ldots \cdot a_n\right) \cdot a_{n+1}\right)^{-1} \\
&=a_{n+1}^{-1} \cdot\left(a_1 \cdot \ldots \cdot a_n\right)^{-1} \\
&=a_{n+1}^{-1} \cdot a_n^{-1} \cdot \ldots \cdot a_1^{-1},
\end{aligned}
$$
using associativity and the base case where necessary.
\end{proof}


\paragraph{Exercise 1.1.16} Let $x$ be an element of $G$. Prove that $x^2=1$ if and only if $|x|$ is either $1$ or $2$.
\begin{proof}
    $(\Rightarrow)$ Suppose $x^2=1$. Then we have $0<|x| \leq 2$, i.e., $|x|$ is either 1 or 2 .
( $\Leftarrow$ ) If $|x|=1$, then we have $x=1$ so that $x^2=1$. If $|x|=2$ then $x^2=1$ by definition. So if $|x|$ is 1 or 2 , we have $x^2=1$.
\end{proof}

\paragraph{Exercise 1.1.17} Let $x$ be an element of $G$. Prove that if $|x|=n$ for some positive integer $n$ then $x^{-1}=x^{n-1}$.
\begin{proof}
    We have $x \cdot x^{n-1}=x^n=1$, so by the uniqueness of inverses $x^{-1}=x^{n-1}$.
\end{proof}


\paragraph{Exercise 1.1.18} Let $x$ and $y$ be elements of $G$. Prove that $xy=yx$ if and only if $y^{-1}xy=x$ if and only if $x^{-1}y^{-1}xy=1$.


\paragraph{Exercise 1.1.20} For $x$ an element in $G$ show that $x$ and $x^{-1}$ have the same order.
\begin{proof}
    Recall that the order of a group element is either a positive integer or infinity.
Suppose $|x|$ is infinite and that $\left|x^{-1}\right|=n$ for some $n$. Then
$$
x^n=x^{(-1) \cdot n \cdot(-1)}=\left(\left(x^{-1}\right)^n\right)^{-1}=1^{-1}=1,
$$
a contradiction. So if $|x|$ is infinite, $\left|x^{-1}\right|$ must also be infinite. Likewise, if $\left|x^{-1}\right|$ is infinite, then $\left|\left(x^{-1}\right)^{-1}\right|=|x|$ is also infinite.
Suppose now that $|x|=n$ and $\left|x^{-1}\right|=m$ are both finite. Then we have
$$
\left(x^{-1}\right)^n=\left(x^n\right)^{-1}=1^{-1}=1,
$$
so that $m \leq n$. Likewise, $n \leq m$. Hence $m=n$ and $x$ and $x^{-1}$ have the same order.
\end{proof}

\paragraph{Exercise 1.1.22a} If $x$ and $g$ are elements of the group $G$, prove that $|x|=\left|g^{-1} x g\right|$.
\begin{proof}
    First we prove a technical lemma:

    {\bf Lemma.} For all $a, b \in G$ and $n \in \mathbb{Z},\left(b^{-1} a b\right)^n=b^{-1} a^n b$.
The statement is clear for $n=0$. We prove the case $n>0$ by induction; the base case $n=1$ is clear. Now suppose $\left(b^{-1} a b\right)^n=b^{-1} a^n b$ for some $n \geq 1$; then
$$
\left(b^{-1} a b\right)^{n+1}=\left(b^{-1} a b\right)\left(b^{-1} a b\right)^n=b^{-1} a b b^{-1} a^n b=b^{-1} a^{n+1} b .
$$
By induction the statement holds for all positive $n$.
Now suppose $n<0$; we have
$$
\left(b^{-1} a b\right)^n=\left(\left(b^{-1} a b\right)^{-n}\right)^{-1}=\left(b^{-1} a^{-n} b\right)^{-1}=b^{-1} a^n b .
$$
Hence, the statement holds for all integers $n$.
Now to the main result. Suppose first that $|x|$ is infinity and that $\left|g^{-1} x g\right|=n$ for some positive integer $n$. Then we have
$$
\left(g^{-1} x g\right)^n=g^{-1} x^n g=1,
$$
and multiplying on the left by $g$ and on the right by $g^{-1}$ gives us that $x^n=1$, a contradiction. Thus if $|x|$ is infinity, so is $\left|g^{-1} x g\right|$. Similarly, if $\left|g^{-1} x g\right|$ is infinite and $|x|=n$, we have
$$
\left(g^{-1} x g\right)^n=g^{-1} x^n g=g^{-1} g=1,
$$
a contradiction. Hence if $\left|g^{-1} x g\right|$ is infinite, so is $|x|$.
Suppose now that $|x|=n$ and $\left|g^{-1} x g\right|=m$ for some positive integers $n$ and $m$. We have
$$
\left(g^{-1} x g\right)^n=g^{-1} x^n g=g^{-1} g=1,
$$
So that $m \leq n$, and
$$
\left(g^{-1} x g\right)^m=g^{-1} x^m g=1,
$$
so that $x^m=1$ and $n \leq m$. Thus $n=m$.
\end{proof}


\paragraph{Exercise 1.1.22b} Deduce that $|a b|=|b a|$ for all $a, b \in G$.
\begin{proof}
    Let $a$ and $b$ be arbitrary group elements. Letting $x=a b$ and $g=a$, we see that
$$
|a b|=\left|a^{-1} a b a\right|=|b a| .
$$
\end{proof}


\paragraph{Exercise 1.1.25} Prove that if $x^{2}=1$ for all $x \in G$ then $G$ is abelian.
\begin{proof}
    Solution: Note that since $x^2=1$ for all $x \in G$, we have $x^{-1}=x$. Now let $a, b \in G$. We have
$$
a b=(a b)^{-1}=b^{-1} a^{-1}=b a .
$$
Thus $G$ is abelian.
\end{proof}


\paragraph{Exercise 1.1.29} Prove that $A \times B$ is an abelian group if and only if both $A$ and $B$ are abelian.
\begin{proof}
    $(\Rightarrow)$ Suppose $a_1, a_2 \in A$ and $b_1, b_2 \in B$. Then
$$
\left(a_1 a_2, b_1 b_2\right)=\left(a_1, b_1\right) \cdot\left(a_2, b_2\right)=\left(a_2, b_2\right) \cdot\left(a_1, b_1\right)=\left(a_2 a_1, b_2 b_1\right) .
$$
Since two pairs are equal precisely when their corresponding entries are equal, we have $a_1 a_2=a_2 a_1$ and $b_1 b_2=b_2 b_1$. Hence $A$ and $B$ are abelian.
$(\Leftarrow)$ Suppose $\left(a_1, b_1\right),\left(a_2, b_2\right) \in A \times B$. Then we have
$$
\left(a_1, b_1\right) \cdot\left(a_2, b_2\right)=\left(a_1 a_2, b_1 b_2\right)=\left(a_2 a_1, b_2 b_1\right)=\left(a_2, b_2\right) \cdot\left(a_1, b_1\right) .
$$
Hence $A \times B$ is abelian.
\end{proof}


\paragraph{Exercise 1.1.34} If $x$ is an element of infinite order in $G$, prove that the elements $x^{n}, n \in \mathbb{Z}$ are all distinct.
\begin{proof}
    Solution: Suppose to the contrary that $x^a=x^b$ for some $0 \leq a<b \leq n-1$. Then we have $x^{b-a}=1$, with $1 \leq b-a<n$. However, recall that $n$ is by definition the least integer $k$ such that $x^k=1$, so we have a contradiction. Thus all the $x^i$, $0 \leq i \leq n-1$, are distinct. In particular, we have
$$
\left\{x^i \mid 0 \leq i \leq n-1\right\} \subseteq G,
$$
so that $|x|=n \leq|G|$
\end{proof}


\paragraph{Exercise 1.3.8} Prove that if $\Omega=\{1,2,3, \ldots\}$ then $S_{\Omega}$ is an infinite group

\paragraph{Exercise 1.6.4} Prove that the multiplicative groups $\mathbb{R}-\{0\}$ and $\mathbb{C}-\{0\}$ are not isomorphic.
\begin{proof}
    Solution: Recall from Exercise 1.6.2 that isomorphic groups necessarily have the same number of elements of order $n$ for all finite $n$.

Now let $x \in \mathbb{R}^{\times}$. If $x=1$ then $|x|=1$, and if $x=-1$ then $|x|=2$. If (with bars denoting absolute value) $|x|<1$, then we have
$$
1>|x|>\left|x^2\right|>\cdots,
$$
and in particular, $1>\left|x^n\right|$ for all $n$. So $x$ has infinite order in $\mathbb{R}^{\times}$.
Similarly, if $|x|>1$ (absolute value) then $x$ has infinite order in $\mathbb{R}^{\times}$. So $\mathbb{R}^{\times}$has 1 element of order 1,1 element of order 2 , and all other elements have infinite order.
In $\mathbb{C}^{\times}$, on the other hand, $i$ has order 4 . Thus $\mathbb{R}^{\times}$and $\mathbb{C}^{\times}$are not isomorphic.
\end{proof}


\paragraph{Exercise 1.6.11} Let $A$ and $B$ be groups. Prove that $A \times B \cong B \times A$.
\begin{proof}
    Solution: We know from set theory that the mapping $\varphi: A \times B \rightarrow B \times A$ given by $\varphi((a, b))=(b, a)$ is a bijection with inverse $\psi: B \times A \rightarrow A \times B$ given by $\psi((b, a))=(a, b)$. Also $\varphi$ is a homomorphism, as we show below.
Let $a_1, a_2 \in A$ and $b_1, b_2 \in B$. Then
$$
\begin{aligned}
\varphi\left(\left(a_1, b_1\right) \cdot\left(a_2, b_2\right)\right) &=\varphi\left(\left(a_1 a_2, b_1 b_2\right)\right) \\
&=\left(b_1 b_2, a_1 a_2\right) \\
&=\left(b_1, a_1\right) \cdot\left(b_2, a_2\right) \\
&=\varphi\left(\left(a_1, b_1\right)\right) \cdot \varphi\left(\left(a_2, b_2\right)\right)
\end{aligned}
$$
Hence $A \times B \cong B \times A$.
\end{proof}


\paragraph{Exercise 1.6.17} Let $G$ be any group. Prove that the map from $G$ to itself defined by $g \mapsto g^{-1}$ is a homomorphism if and only if $G$ is abelian.
\begin{proof}
    $(\Rightarrow)$ Suppose $G$ is abelian. Then
$$
\varphi(a b)=(a b)^{-1}=b^{-1} a^{-1}=a^{-1} b^{-1}=\varphi(a) \varphi(b),
$$
so that $\varphi$ is a homomorphism.
$(\Leftarrow)$ Suppose $\varphi$ is a homomorphism, and let $a, b \in G$. Then
$$
a b=\left(b^{-1} a^{-1}\right)^{-1}=\varphi\left(b^{-1} a^{-1}\right)=\varphi\left(b^{-1}\right) \varphi\left(a^{-1}\right)=\left(b^{-1}\right)^{-1}\left(a^{-1}\right)^{-1}=b a,
$$
so that $G$ is abelian.
\end{proof}


\paragraph{Exercise 1.6.23} Let $G$ be a finite group which possesses an automorphism $\sigma$ such that $\sigma(g)=g$ if and only if $g=1$. If $\sigma^{2}$ is the identity map from $G$ to $G$, prove that $G$ is abelian.
\begin{proof}
    Solution: We define a mapping $f: G \rightarrow G$ by $f(x)=x^{-1} \sigma(x)$.
Claim: $f$ is injective.
Proof of claim: Suppose $f(x)=f(y)$. Then $y^{-1} \sigma(y)=x^{-1} \sigma(x)$, so that $x y^{-1}=\sigma(x) \sigma\left(y^{-1}\right)$, and $x y^{-1}=\sigma\left(x y^{-1}\right)$. Then we have $x y^{-1}=1$, hence $x=y$. So $f$ is injective.

Since $G$ is finite and $f$ is injective, $f$ is also surjective. Then every $z \in G$ is of the form $x^{-1} \sigma(x)$ for some $x$. Now let $z \in G$ with $z=x^{-1} \sigma(x)$. We have
$$
\sigma(z)=\sigma\left(x^{-1} \sigma(x)\right)=\sigma(x)^{-1} x=\left(x^{-1} \sigma(x)\right)^{-1}=z^{-1} .
$$
Thus $\sigma$ is in fact the inversion mapping, and we assumed that $\sigma$ is a homomorphism. By a previous example, then, $G$ is abelian.
\end{proof}


\paragraph{Exercise 1.7.5} Prove that the kernel of an action of the group $G$ on a set $A$ is the same as the kernel of the corresponding permutation representation $G\to S_A$.
\begin{proof}
    Solution: Let $G$ be a group acting on $A$. The kernel of the action is the set
$$
K=\{g \in G \mid g \cdot a=a \text { for all } a \in A\} .
$$
The corresponding permutation representation is a group homomorphism $\varphi: G \rightarrow S_A$ given by $\varphi(g)(a)=g \cdot a$, and by definition
$$
\operatorname{ker} \varphi=\{g \in G \mid \varphi(g)=1\} \text {. }
$$
$K \subseteq \operatorname{ker} \varphi$ : Let $k \in K$. Then for all $a \in A$, we have
$$
\varphi(k)(a)=k \cdot a=a,
$$
so that
$$
\varphi(k)=\mathrm{id}_A=1 .
$$
Thus $g \in \operatorname{ker} \varphi$.
$\operatorname{ker} \varphi \subseteq K$ : Let $k \in \operatorname{ker} \varphi$. Then for all $a \in A$, we have
$$
k \cdot a=\varphi(k)(a)=\mathrm{id}_A(a)=a .
$$
Thus $k \in K$.
\end{proof}


\paragraph{Exercise 1.7.6} Prove that a group $G$ acts faithfully on a set $A$ if and only if the kernel of the action is the set consisting only of the identity.
\begin{proof}
    Solution: We know that a group action is faithful precisely when the corresponding permutation representation $\varphi: G \rightarrow S_A$ is injective. Moreover, a group homomorphism is injective precisely when its kernel is trivial. The kernel of a group action is equal to the kernel of the corresponding permutation representation. So $G$ acts faithfully on $A$ if and only if the kernel of the action is trivial.
\end{proof}


\paragraph{Exercise 2.1.5} Prove that $G$ cannot have a subgroup $H$ with $|H|=n-1$, where $n=|G|>2$.
\begin{proof}
    Solution: Under these conditions, there exists a nonidentity element $x \in H$ and an element $y \notin H$. Consider the product $x y$. If $x y \in H$, then since $x^{-1} \in H$ and $H$ is a subgroup, $y \in H$, a contradiction. If $x y \notin H$, then we have $x y=y$. Thus $x=1$, a contradiction. Thus no such subgroup exists.
\end{proof}


\paragraph{Exercise 2.1.13} Let $H$ be a subgroup of the additive group of rational numbers with the property that $1 / x \in H$ for every nonzero element $x$ of $H$. Prove that $H=0$ or $\mathbb{Q}$.
\begin{proof}
    Solution: First, suppose there does not exist a nonzero element in $H$. Then $H=0$.
Now suppose there does exist a nonzero element $a \in H$; without loss of generality, say $a=p / q$ in lowest terms for some integers $p$ and $q$ - that is, $\operatorname{gcd}(p, q)=1$. Now $q \cdot \frac{p}{q}=p \in H$, and since $q / p \in H$, we have $p \cdot \frac{q}{p} \in H$. There exist integers $x, y$ such that $q x+p y=1$; note that $q x \in H$ and $p y \in H$, so that $1 \in H$. Thus $n \in H$ for all $n \in \mathbb{Z}$. Moreover, if $n \neq 0,1 / n \in H$. Then $m / n \in H$ for all integers $m, n$ with $n \neq 0$; hence $H=\mathbb{Q}$.
\end{proof}


\paragraph{Exercise 2.4.4} Prove that if $H$ is a subgroup of $G$ then $H$ is generated by the set $H-\{1\}$.

\paragraph{Exercise 2.4.13} Prove that the multiplicative group of positive rational numbers is generated by the set $\left\{\frac{1}{p} \mid \text{$p$ is a prime} \right\}$.

\paragraph{Exercise 2.4.16a} A subgroup $M$ of a group $G$ is called a maximal subgroup if $M \neq G$ and the only subgroups of $G$ which contain $M$ are $M$ and $G$. Prove that if $H$ is a proper subgroup of the finite group $G$ then there is a maximal subgroup of $G$ containing $H$.

\paragraph{Exercise 2.4.16b} Show that the subgroup of all rotations in a dihedral group is a maximal subgroup.

\paragraph{Exercise 2.4.16c} Show that if $G=\langle x\rangle$ is a cyclic group of order $n \geq 1$ then a subgroup $H$ is maximal if and only $H=\left\langle x^{p}\right\rangle$ for some prime $p$ dividing $n$.

\paragraph{Exercise 3.1.3a} Let $A$ be an abelian group and let $B$ be a subgroup of $A$. Prove that $A / B$ is abelian.
\begin{proof}
    Lemma: Let $G$ be a group. If $|G|=2$, then $G \cong Z_2$.
Proof: Since $G=\{e a\}$ has an identity element, say $e$, we know that $e e=e, e a=a$, and $a e=a$. If $a^2=a$, we have $a=e$, a contradiction. Thus $a^2=e$. We can easily see that $G \cong Z_2$.

If $A$ is abelian, every subgroup of $A$ is normal; in particular, $B$ is normal, so $A / B$ is a group. Now let $x B, y B \in A / B$. Then
$$
(x B)(y B)=(x y) B=(y x) B=(y B)(x B) .
$$
Hence $A / B$ is abelian.
\end{proof}


\paragraph{Exercise 3.1.22a} Prove that if $H$ and $K$ are normal subgroups of a group $G$ then their intersection $H \cap K$ is also a normal subgroup of $G$.

\paragraph{Exercise 3.1.22b} Prove that the intersection of an arbitrary nonempty collection of normal subgroups of a group is a normal subgroup (do not assume the collection is countable).

\paragraph{Exercise 3.2.8} Prove that if $H$ and $K$ are finite subgroups of $G$ whose orders are relatively prime then $H \cap K=1$.
\begin{proof}
    Solution: Let $|H|=p$ and $|K|=q$. We saw in a previous exercise that $H \cap K$ is a subgroup of both $H$ and $K$; by Lagrange's Theorem, then, $|H \cap K|$ divides $p$ and $q$. Since $\operatorname{gcd}(p, q)=1$, then, $|H \cap K|=1$. Thus $H \cap K=1$.
\end{proof}


\paragraph{Exercise 3.2.11} Let $H \leq K \leq G$. Prove that $|G: H|=|G: K| \cdot|K: H|$ (do not assume $G$ is finite).

\paragraph{Exercise 3.2.16} Use Lagrange's Theorem in the multiplicative group $(\mathbb{Z} / p \mathbb{Z})^{\times}$to prove Fermat's Little Theorem: if $p$ is a prime then $a^{p} \equiv a(\bmod p)$ for all $a \in \mathbb{Z}$.
\begin{proof}
    Solution: If $p$ is prime, then $\varphi(p)=p-1$ (where $\varphi$ denotes the Euler totient). Thus
$$
\mid\left((\mathbb{Z} /(p))^{\times} \mid=p-1 .\right.
$$
So for all $a \in(\mathbb{Z} /(p))^{\times}$, we have $|a|$ divides $p-1$. Hence
$$
a=1 \cdot a=a^{p-1} a=a^p \quad(\bmod p) .
$$
\end{proof}


\paragraph{Exercise 3.2.21a} Prove that $\mathbb{Q}$ has no proper subgroups of finite index.
\begin{proof}
    Solution: We begin with a lemma.
Lemma: If $D$ is a divisible abelian group, then no proper subgroup of $D$ has finite index.
Proof: We saw previously that no finite group is divisible and that every proper quotient $D / A$ of a divisible group is divisible; thus no proper quotient of a divisible group is finite. Equivalently, $[D: A]$ is not finite.
Because $\mathbb{Q}$ and $\mathbb{Q} / \mathbb{Z}$ are divisible, the conclusion follows.
\end{proof}


\paragraph{Exercise 3.3.3} Prove that if $H$ is a normal subgroup of $G$ of prime index $p$ then for all $K \leq G$ either $K \leq H$, or $G=H K$ and $|K: K \cap H|=p$.
\begin{proof}
    Solution: Suppose $K \backslash N \neq \emptyset$; say $k \in K \backslash N$. Now $G / N \cong \mathbb{Z} /(p)$ is cyclic, and moreover is generated by any nonidentity- in particular by $\bar{k}$

Now $K N \leq G$ since $N$ is normal. Let $g \in G$. We have $g N=k^a N$ for some integer a. In particular, $g=k^a n$ for some $n \in N$, hence $g \in K N$. We have $[K: K \cap N]=p$ by the Second Isomorphism Theorem.
\end{proof}


\paragraph{Exercise 3.4.1} Prove that if $G$ is an abelian simple group then $G \cong Z_{p}$ for some prime $p$ (do not assume $G$ is a finite group).
\begin{proof}
    Solution: Let $G$ be an abelian simple group.
Suppose $G$ is infinite. If $x \in G$ is a nonidentity element of finite order, then $\langle x\rangle<G$ is a nontrivial normal subgroup, hence $G$ is not simple. If $x \in G$ is an element of infinite order, then $\left\langle x^2\right\rangle$ is a nontrivial normal subgroup, so $G$ is not simple.

Suppose $G$ is finite; say $|G|=n$. If $n$ is composite, say $n=p m$ for some prime $p$ with $m \neq 1$, then by Cauchy's Theorem $G$ contains an element $x$ of order $p$ and $\langle x\rangle$ is a nontrivial normal subgroup. Hence $G$ is not simple. Thus if $G$ is an abelian simple group, then $|G|=p$ is prime. We saw previously that the only such group up to isomorphism is $\mathbb{Z} /(p)$, so that $G \cong \mathbb{Z} /(p)$. Moreover, these groups are indeed simple.
\end{proof}


\paragraph{Exercise 3.4.4} Use Cauchy's Theorem and induction to show that a finite abelian group has a subgroup of order $n$ for each positive divisor $n$ of its order.

\paragraph{Exercise 3.4.5a} Prove that subgroups of a solvable group are solvable.

\paragraph{Exercise 3.4.5b} Prove that quotient groups of a solvable group are solvable.

\paragraph{Exercise 3.4.11} Prove that if $H$ is a nontrivial normal subgroup of the solvable group $G$ then there is a nontrivial subgroup $A$ of $H$ with $A \unlhd G$ and $A$ abelian.

\paragraph{Exercise 4.2.8} Prove that if $H$ has finite index $n$ then there is a normal subgroup $K$ of $G$ with $K \leq H$ and $|G: K| \leq n!$.
\begin{proof}
    Solution: $G$ acts on the cosets $G / H$ by left multiplication. Let $\lambda: G \rightarrow S_{G / H}$ be the permutation representation induced by this action, and let $K$ be the kernel of the representation.
Now $K$ is normal in $G$, and $K \leq \operatorname{stab}_G(H)=H$. By the First Isomorphism Theorem, we have an injective group homomorphism $\bar{\lambda}: G / K \rightarrow S_{G / H}$. Since $\left|S_{G / H}\right|=n !$, we have $[G: K] \leq n !$.
\end{proof}


\paragraph{Exercise 4.2.9a} Prove that if $p$ is a prime and $G$ is a group of order $p^{\alpha}$ for some $\alpha \in \mathbb{Z}^{+}$, then every subgroup of index $p$ is normal in $G$.
\begin{proof}
    Solution: Let $G$ be a group of order $p^k$ and $H \leq G$ a subgroup with $[G: H]=p$. Now $G$ acts on the conjugates $g H g^{-1}$ by conjugation, since
$$
g_1 g_2 \cdot H=\left(g_1 g_2\right) H\left(g_1 g_2\right)^{-1}=g_1\left(g_2 H g_2^{-1}\right) g_1^{-1}=g_1 \cdot\left(g_2 \cdot H\right)
$$
and $1 \cdot H=1 H 1=H$. Moreover, under this action we have $H \leq \operatorname{stab}(H)$. By Exercise 3.2.11, we have
$$
[G: \operatorname{stab}(H)][\operatorname{stab}(H): H]=[G: H]=p,
$$
a prime.
If $[G: \operatorname{stab}(H)]=p$, then $[\operatorname{stab}(H): H]=1$ and we have $H=\operatorname{stab}(H)$; moreover, $H$ has exactly $p$ conjugates in $G$. Let $\varphi: G \rightarrow S_p$ be the permutation representation induced by the action of $G$ on the conjugates of $H$, and let $K$ be the kernel of this representation. Now $K \leq \operatorname{stab}(H)=H$. By the first isomorphism theorem, the induced map $\bar{\varphi}: G / K \rightarrow S_p$ is injective, so that $|G / K|$ divides $p$ !. Note, however, that $|G / K|$ is a power of $p$ and that the only powers of $p$ that divide $p$ ! are 1 and $p$. So $[G: K]$ is 1 or $p$. If $[G: K]=1$, then $G=K$ so that $g H g^{-1}=H$ for all $g \in G$; then $\operatorname{stab}(H)=G$ and we have $[G: \operatorname{stab}(H)]=1$, a contradiction. Now suppose $[G: K]=p$. Again by Exercise $3.2$.11 we have $[G: K]=[G: H][H: K]$, so that $[H: K]=1$, hence $H=K$. Again, this implies that $H$ is normal so that $g H g^{-1}=H$ for all $g \in G$, and we have $[G: \operatorname{stab}(H)]=1$, a contradiction. Thus $[G: \operatorname{stab}(H)] \neq p$
If $[G: \operatorname{stab}(H)]=1$, then $G=\operatorname{stab}(H)$. That is, $g H g^{-1}=H$ for all $g \in G$; thus $H \leq G$ is normal.
\end{proof}


\paragraph{Exercise 4.2.14} Let $G$ be a finite group of composite order $n$ with the property that $G$ has a subgroup of order $k$ for each positive integer $k$ dividing $n$. Prove that $G$ is not simple.
\begin{proof}
    Solution: Let $p$ be the smallest prime dividing $n$, and write $n=p m$. Now $G$ has a subgroup $H$ of order $m$, and $H$ has index $p$. By Corollary 5 in the text, $H$ is normal in $G$.
\end{proof}


\paragraph{Exercise 4.3.5} If the center of $G$ is of index $n$, prove that every conjugacy class has at most $n$ elements.

\paragraph{Exercise 4.3.26} Let $G$ be a transitive permutation group on the finite set $A$ with $|A|>1$. Show that there is some $\sigma \in G$ such that $\sigma(a) \neq a$ for all $a \in A$.

\paragraph{Exercise 4.3.27} Let $g_{1}, g_{2}, \ldots, g_{r}$ be representatives of the conjugacy classes of the finite group $G$ and assume these elements pairwise commute. Prove that $G$ is abelian.

\paragraph{Exercise 4.4.2} Prove that if $G$ is an abelian group of order $p q$, where $p$ and $q$ are distinct primes, then $G$ is cyclic.

\paragraph{Exercise 4.4.6a} Prove that characteristic subgroups are normal.

\paragraph{Exercise 4.4.6b} Prove that there exists a normal subgroup that is not characteristic.

\paragraph{Exercise 4.4.7} If $H$ is the unique subgroup of a given order in a group $G$ prove $H$ is characteristic in $G$.

\paragraph{Exercise 4.4.8a} Let $G$ be a group with subgroups $H$ and $K$ with $H \leq K$. Prove that if $H$ is characteristic in $K$ and $K$ is normal in $G$ then $H$ is normal in $G$.

\paragraph{Exercise 4.5.1a} Prove that if $P \in \operatorname{Syl}_{p}(G)$ and $H$ is a subgroup of $G$ containing $P$ then $P \in \operatorname{Syl}_{p}(H)$.
\begin{proof}
    Solution: If $P \leq H \leq G$ is a Sylow $p$-subgroup of $G$, then $p$ does not divide $[G: P]$. Now $[G: P]=[G: H][H: P]$, so that $p$ does not divide $[H: P]$; hence $P$ is a Sylow $p$-subgroup of $H$.
\end{proof}


\paragraph{Exercise 4.5.13} Prove that a group of order 56 has a normal Sylow $p$-subgroup for some prime $p$ dividing its order.

\paragraph{Exercise 4.5.14} Prove that a group of order 312 has a normal Sylow $p$-subgroup for some prime $p$ dividing its order.

\paragraph{Exercise 4.5.15} Prove that a group of order 351 has a normal Sylow $p$-subgroup for some prime $p$ dividing its order.

\paragraph{Exercise 4.5.16} Let $|G|=p q r$, where $p, q$ and $r$ are primes with $p<q<r$. Prove that $G$ has a normal Sylow subgroup for either $p, q$ or $r$.

\paragraph{Exercise 4.5.17} Prove that if $|G|=105$ then $G$ has a normal Sylow 5 -subgroup and a normal Sylow 7-subgroup.

\paragraph{Exercise 4.5.18} Prove that a group of order 200 has a normal Sylow 5-subgroup.

\paragraph{Exercise 4.5.19} Prove that if $|G|=6545$ then $G$ is not simple.

\paragraph{Exercise 4.5.20} Prove that if $|G|=1365$ then $G$ is not simple.

\paragraph{Exercise 4.5.21} Prove that if $|G|=2907$ then $G$ is not simple.

\paragraph{Exercise 4.5.22} Prove that if $|G|=132$ then $G$ is not simple.

\paragraph{Exercise 4.5.23} Prove that if $|G|=462$ then $G$ is not simple.

\paragraph{Exercise 4.5.28} Let $G$ be a group of order 105. Prove that if a Sylow 3-subgroup of $G$ is normal then $G$ is abelian.

\paragraph{Exercise 4.5.33} Let $P$ be a normal Sylow $p$-subgroup of $G$ and let $H$ be any subgroup of $G$. Prove that $P \cap H$ is the unique Sylow $p$-subgroup of $H$.

\paragraph{Exercise 5.4.2} Prove that a subgroup $H$ of $G$ is normal if and only if $[G, H] \leq H$.

\paragraph{Exercise 7.1.2} Prove that if $u$ is a unit in $R$ then so is $-u$.
\begin{proof}
    Solution: Since $u$ is a unit, we have $u v=v u=1$ for some $v \in R$. Thus, we have
$$
(-v)(-u)=v u=1
$$
and
$$
(-u)(-v)=u v=1 .
$$
Thus $-u$ is a unit.
\end{proof}


\paragraph{Exercise 7.1.11} Prove that if $R$ is an integral domain and $x^{2}=1$ for some $x \in R$ then $x=\pm 1$.
\begin{proof}
    Solution: If $x^2=1$, then $x^2-1=0$. Evidently, then,
$$
(x-1)(x+1)=0 .
$$
Since $R$ is an integral domain, we must have $x-1=0$ or $x+1=0$; thus $x=1$ or $x=-1$.
\end{proof}


\paragraph{Exercise 7.1.12} Prove that any subring of a field which contains the identity is an integral domain.
\begin{proof}
    Solution: Let $R \subseteq F$ be a subring of a field. (We need not yet assume that $1 \in R$ ). Suppose $x, y \in R$ with $x y=0$. Since $x, y \in F$ and the zero element in $R$ is the same as that in $F$, either $x=0$ or $y=0$. Thus $R$ has no zero divisors. If $R$ also contains 1 , then $R$ is an integral domain.
\end{proof}


\paragraph{Exercise 7.1.15} A ring $R$ is called a Boolean ring if $a^{2}=a$ for all $a \in R$. Prove that every Boolean ring is commutative.
\begin{proof}
    Solution: Note first that for all $a \in R$,
$$
-a=(-a)^2=(-1)^2 a^2=a^2=a .
$$
Now if $a, b \in R$, we have
$$
a+b=(a+b)^2=a^2+a b+b a+b^2=a+a b+b a+b .
$$
Thus $a b+b a=0$, and we have $a b=-b a$. But then $a b=b a$. Thus $R$ is commutative.
\end{proof}


\paragraph{Exercise 7.2.2} Let $p(x)=a_{n} x^{n}+a_{n-1} x^{n-1}+\cdots+a_{1} x+a_{0}$ be an element of the polynomial ring $R[x]$. Prove that $p(x)$ is a zero divisor in $R[x]$ if and only if there is a nonzero $b \in R$ such that $b p(x)=0$.
\begin{proof}
    Solution: If $b p(x)=0$ for some nonzero $b \in R$, then it is clear that $p(x)$ is a zero divisor.
Now suppose $p(x)$ is a zero divisor; that is, for some $q(x)=\sum_{i=0}^m b_i x^i$, we have $p(x) q(x)=0$. We may choose $q(x)$ to have minimal degree among the nonzero polynomials with this property.
We will now show by induction that $a_i q(x)=0$ for all $0 \leq i \leq n$.
For the base case, note that
$$
p(x) q(x)=\sum_{k=0}^{n+m}\left(\sum_{i+j=k} a_i b_j\right) x^k=0 .
$$
The coefficient of $x^{n+m}$ in this product is $a_n b_m$ on one hand, and 0 on the other. Thus $a_n b_m=0$. Now $a_n q(x) p(x)=0$, and the coefficient of $x^m$ in $q$ is $a_n b_m=0$. Thus the degree of $a_n q(x)$ is strictly less than that of $q(x)$; since $q(x)$ has minimal degree among the nonzero polynomials which multiply $p(x)$ to 0 , in fact $a_n q(x)=0$. More specifically, $a_n b_i=0$ for all $0 \leq i \leq m$.
For the inductive step, suppose that for some $0 \leq t<n$, we have $a_r q(x)=0$ for all $t<r \leq n$. Now
$$
p(x) q(x)=\sum_{k=0}^{n+m}\left(\sum_{i+j=k} a_i b_j\right) x^k=0 .
$$
On one hand, the coefficient of $x^{m+t}$ is $\sum_{i+j=m+t} a_i b_j$, and on the other hand, it is 0 . Thus
$$
\sum_{i+j=m+t} a_i b_j=0 .
$$
By the induction hypothesis, if $i \geq t$, then $a_i b_j=0$. Thus all terms such that $i \geq t$ are zero. If $i<t$, then we must have $j>m$, a contradiction. Thus we have $a_t b_m=0$. As in the base case,
$$
a_t q(x) p(x)=0
$$
and $a_t q(x)$ has degree strictly less than that of $q(x)$, so that by minimality, $a_t q(x)=0$.
By induction, $a_i q(x)=0$ for all $0 \leq i \leq n$. In particular, $a_i b_m=0$. Thus $b_m p(x)=0$.
\end{proof}

\paragraph{Exercise 7.2.12} Let $G=\left\{g_{1}, \ldots, g_{n}\right\}$ be a finite group. Prove that the element $N=g_{1}+g_{2}+\ldots+g_{n}$ is in the center of the group ring $R G$.
\begin{proof}
    Solution: Let $M=\sum_{i=1}^n r_i g_i$ be an element of $R[G]$. Note that for each $g_i \in G$, the action of $g_i$ on $G$ by conjugation permutes the subscripts. Then we have the following.
$$
\begin{aligned}
N M &=\left(\sum_{i=1}^n g_i\right)\left(\sum_{j=1}^n r_j g_j\right) \\
&=\sum_{j=1}^n \sum_{i=1}^n r_j g_i g_j \\
&=\sum_{j=1}^n \sum_{i=1}^n r_j g_j g_j^{-1} g_i g_j \\
&=\sum_{j=1}^n r_j g_j\left(\sum_{i=1}^n g_j^{-1} g_i g_j\right) \\
&=\sum_{j=1}^n r_j g_j\left(\sum_{i=1}^n g_i\right) \\
&=\left(\sum_{j=1}^n r_j g_j\right)\left(\sum_{i=1}^n g_i\right) \\
&=M N .
\end{aligned}
$$
Thus $N \in Z(R[G])$.
\end{proof}


\paragraph{Exercise 7.3.16} Let $\varphi: R \rightarrow S$ be a surjective homomorphism of rings. Prove that the image of the center of $R$ is contained in the center of $S$.
\begin{proof}
    Solution: Suppose $r \in \varphi[Z(R)]$. Then $r=\varphi(z)$ for some $z \in Z(R)$. Now let $x \in S$. Since $\varphi$ is surjective, we have $x=\varphi y$ for some $y \in R$. Now
$$
x r=\varphi(y) \varphi(z)=\varphi(y z)=\varphi(z y)=\varphi(z) \varphi(y)=r x .
$$
Thus $r \in Z(S)$.
\end{proof}

\paragraph{Exercise 7.3.37} An ideal $N$ is called nilpotent if $N^{n}$ is the zero ideal for some $n \geq 1$. Prove that the ideal $p \mathbb{Z} / p^{m} \mathbb{Z}$ is a nilpotent ideal in the ring $\mathbb{Z} / p^{m} \mathbb{Z}$.
\begin{proof}
    Solution: First we prove a lemma.
Lemma: Let $R$ be a ring, and let $I_1, I_2, J \subseteq R$ be ideals such that $J \subseteq I_1, I_2$. Then $\left(I_1 / J\right)\left(I_2 / J\right)=I_1 I_2 / J$.
Proof: ( $\subseteq$ ) Let
$$
\alpha=\sum\left(x_i+J\right)\left(y_i+J\right) \in\left(I_1 / J\right)\left(I_2 / J\right) .
$$
Then
$$
\alpha=\sum\left(x_i y_i+J\right)=\left(\sum x_i y_i\right)+J \in\left(I_1 I_2\right) / J .
$$
Now let $\alpha=\left(\sum x_i y_i\right)+J \in\left(I_1 I_2\right) / J$. Then
$$
\alpha=\sum\left(x_i+J\right)\left(y_i+J\right) \in\left(I_1 / J\right)\left(I_2 / J\right) .
$$
From this lemma and the lemma to Exercise 7.3.36, it follows by an easy induction that
$$
\left(p \mathbb{Z} / p^m \mathbb{Z}\right)^m=(p \mathbb{Z})^m / p^m \mathbb{Z}=p^m \mathbb{Z} / p^m \mathbb{Z} \cong 0 .
$$
Thus $p \mathbb{Z} / p^m \mathbb{Z}$ is nilpotent in $\mathbb{Z} / p^m \mathbb{Z}$.
\end{proof}


\paragraph{Exercise 7.4.27} Let $R$ be a commutative ring with $1 \neq 0$. Prove that if $a$ is a nilpotent element of $R$ then $1-a b$ is a unit for all $b \in R$.
\begin{proof}
    $\mathfrak{N}(R)$ is an ideal of $R$. Thus for all $b \in R,-a b$ is nilpotent. Hence $1-a b$ is a unit in $R$.
\end{proof}


\paragraph{Exercise 8.1.12} Let $N$ be a positive integer. Let $M$ be an integer relatively prime to $N$ and let $d$ be an integer relatively prime to $\varphi(N)$, where $\varphi$ denotes Euler's $\varphi$-function. Prove that if $M_{1} \equiv M^{d} \pmod N$ then $M \equiv M_{1}^{d^{\prime}} \pmod N$ where $d^{\prime}$ is the inverse of $d \bmod \varphi(N)$: $d d^{\prime} \equiv 1 \pmod {\varphi(N)}$.

\paragraph{Exercise 8.2.4} Let $R$ be an integral domain. Prove that if the following two conditions hold then $R$ is a Principal Ideal Domain: (i) any two nonzero elements $a$ and $b$ in $R$ have a greatest common divisor which can be written in the form $r a+s b$ for some $r, s \in R$, and (ii) if $a_{1}, a_{2}, a_{3}, \ldots$ are nonzero elements of $R$ such that $a_{i+1} \mid a_{i}$ for all $i$, then there is a positive integer $N$ such that $a_{n}$ is a unit times $a_{N}$ for all $n \geq N$.

\paragraph{Exercise 8.3.4} Prove that if an integer is the sum of two rational squares, then it is the sum of two integer squares.

\paragraph{Exercise 8.3.5a} Let $R=\mathbb{Z}[\sqrt{-n}]$ where $n$ is a squarefree integer greater than 3. Prove that $2, \sqrt{-n}$ and $1+\sqrt{-n}$ are irreducibles in $R$.

\paragraph{Exercise 8.3.6a} Prove that the quotient ring $\mathbb{Z}[i] /(1+i)$ is a field of order 2.

\paragraph{Exercise 8.3.6b} Let $q \in \mathbb{Z}$ be a prime with $q \equiv 3 \bmod 4$. Prove that the quotient ring $\mathbb{Z}[i] /(q)$ is a field with $q^{2}$ elements.

\paragraph{Exercise 9.1.6} Prove that $(x, y)$ is not a principal ideal in $\mathbb{Q}[x, y]$.

\paragraph{Exercise 9.1.10} Prove that the ring $\mathbb{Z}\left[x_{1}, x_{2}, x_{3}, \ldots\right] /\left(x_{1} x_{2}, x_{3} x_{4}, x_{5} x_{6}, \ldots\right)$ contains infinitely many minimal prime ideals (cf. exercise.9.1.36 of Section 7.4).

\paragraph{Exercise 9.3.2} Prove that if $f(x)$ and $g(x)$ are polynomials with rational coefficients whose product $f(x) g(x)$ has integer coefficients, then the product of any coefficient of $g(x)$ with any coefficient of $f(x)$ is an integer.

\paragraph{Exercise 9.4.2a} Prove that $x^4-4x^3+6$ is irreducible in $\mathbb{Z}[x]$.

\paragraph{Exercise 9.4.2b} Prove that $x^6+30x^5-15x^3 + 6x-120$ is irreducible in $\mathbb{Z}[x]$.

\paragraph{Exercise 9.4.2c} Prove that $x^4+4x^3+6x^2+2x+1$ is irreducible in $\mathbb{Z}[x]$.

\paragraph{Exercise 9.4.2d} Prove that $\frac{(x+2)^p-2^p}{x}$, where $p$ is an odd prime, is irreducible in $\mathbb{Z}[x]$.

\paragraph{Exercise 9.4.9} Prove that the polynomial $x^{2}-\sqrt{2}$ is irreducible over $\mathbb{Z}[\sqrt{2}]$. You may assume that $\mathbb{Z}[\sqrt{2}]$ is a U.F.D.

\paragraph{Exercise 9.4.11} Prove that $x^2+y^2-1$ is irreducible in $\mathbb{Q}[x,y]$.

\paragraph{Exercise 11.1.13} Prove that as vector spaces over $\mathbb{Q}, \mathbb{R}^n \cong \mathbb{R}$, for all $n \in \mathbb{Z}^{+}$.

\end{document}
